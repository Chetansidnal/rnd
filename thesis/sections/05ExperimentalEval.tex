\section{Experimental Evaluation}
\label{sec:exp_eval}
[TODO]

\subsection{Hardware Setup}
\label{exp_eval:hardware}
We used a desktop computer with the following specifications:
\begin{itemize}
	\item Gigabyte GA-X99-SLI Intel X99 (motherboard)
	\item Intel Xeon E5-1620 V4 4x 3.50GHz (CPU)
	\item 32GB DDR4-2133 DIMM (RAM) 
	\item 8GB MSI GeForce GTX 1080 (GPU)
\end{itemize}

\subsection{Evaluation Criteria}
\label{exp_eval:eval_criteria}

	
\subsection{Experiments}
\label{exp_eval:experiments}


\subsubsection{Experiment One}
\label{experiment:experiment_one}


%\begin{figure}[H]
%	\centering
%	\begin{subfigure}[b]{1.8in}
%		\centering
%		\includegraphics[width=1.8in]{images/05_experiments/gaussian/0.png}
%		\caption{Epoch 1}
%	\end{subfigure}
%	\quad
%	\begin{subfigure}[b]{1.8in}
%		\centering
%		\includegraphics[width=1.8in]{images/05_experiments/gaussian/2600.png}
%		\caption{Epoch 2600}
%	\end{subfigure}
%	\quad
%	\begin{subfigure}[b]{1.8in}
%		\centering
%		\includegraphics[width=1.8in]{images/05_experiments/gaussian/4600.png}
%		\caption{Epoch 4600}
%	\end{subfigure}
%	
%	\vskip\baselineskip
%	
%	\begin{subfigure}[b]{1.8in}
%		\centering
%		\includegraphics[width=1.8in]{images/05_experiments/gaussian/5200.png}
%		\caption{Epoch 5200}
%	\end{subfigure}
%	\quad
%	\begin{subfigure}[b]{1.8in}
%		\centering
%		\includegraphics[width=1.8in]{images/05_experiments/gaussian/5400.png}
%		\caption{Epoch 5400}
%	\end{subfigure}
%	\quad
%	\begin{subfigure}[b]{1.8in}
%		\centering
%		\includegraphics[width=1.8in]{images/05_experiments/gaussian/6200.png}
%		\caption{Epoch 6200}
%	\end{subfigure}
%	
%	\vskip\baselineskip
%	
%	\begin{subfigure}[b]{1.8in}
%		\centering
%		\includegraphics[width=1.8in]{images/05_experiments/gaussian/6800.png}
%		\caption{Epoch 6800}
%	\end{subfigure}
%	\quad
%	\begin{subfigure}[b]{1.8in}
%		\centering
%		\includegraphics[width=1.8in]{images/05_experiments/gaussian/10400.png}
%		\caption{Epoch 10400}
%	\end{subfigure}
%	\quad
%	\begin{subfigure}[b]{1.8in}
%		\centering
%		\includegraphics[width=1.8in]{images/05_experiments/gaussian/13000.png}
%		\caption{Epoch 13000}
%	\end{subfigure}
%	
%	\caption{Generated Gaussian using a simple GAN.} 
%	\raggedright
%	The blue line is the underlying data distribution $p_{data}$ which is a Gaussian $\mathcal{N}(0,1)$. 10000 samples $x$ are taken from $p_{data}$ and the density function is plotted. The red line is the output of the generator $G$. 10000 samples $G(z)$ are taken from the generator's distribution $p_{g}$ at every 100 Epochs/iterations and its density function is plotted. At Epoch 1, nearly all the values that the generator outputs are around 0. It is only around Epoch 2000 that the loss of the generator and discriminator stabilize enough that the generator consistently learns to generate values with more variance. At around Epoch 6000 the generated values start to resemble the underlying distribution but it takes a further 4000 epochs to fully resemble a Gaussian distribution.
%	\label{exp_fig:simple_gan_gaussian}
%\end{figure}

